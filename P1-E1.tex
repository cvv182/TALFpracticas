\documentclass{article}
\usepackage[utf8]{inputenc}
\usepackage{amsthm, amsmath}
\title{Práctica 1 - Ejercicio 1}
\author{Carlos Velasco Hilario}
\date{October 2022}
\usepackage{amsmath}
\usepackage{graphicx}
\begin{document}
\maketitle{}

\section*{Enunciado}
Find the power set R³ of R = {(1, 1), (1, 2), (2, 3), (3, 4)}. Check your answer with the script powerrelation.m and write a LATEX document with the
solution step by step.
\\
\section*{Resultado}
\begin{equation}
R^n =
    \begin{cases}
   
            R & n = 1
          \\  \{(a,b) : \exists x \in A, (a,x) \in R^{n-1} \wedge (x,b) \in R\} & n > 1
           
   
    \end{cases}
\end{equation}

\section*{Resolución}
$R$ es el conjunto inicial con el que empezamos el ejercicio el cual esta formado por $\lbrace{(1, 1),(1, 2),(2, 3),(3, 4)}\rbrace$. Usaremos la definición dada en el apartado anterior para así afirmar que $R^2$ está formado por este conjunto de pares $\lbrace{(1, 1),(1, 2),(1, 3),(2, 4)}\rbrace$.Dado que el conjunto de $R^3$ se crea a partir de los pares $(a,b)$ en el cual el par $(a,x)$ es perteneciente de $R^2$  y$(x,b)$ pertenece a $R $.
\\
\\
De esta manera obtenemos el conjunto de $R^3$: $R^3 = \lbrace{(1,1),(1,2),(1,3),(1,4)}\rbrace$ el cual coincide con el resultado que he obtenido a partir de la función del script de powerrelation.m

\section*{Desarrollo de la resolución}
$R =\lbrace{(1, 1),(1, 2),(2, 3),(3, 4)}\rbrace$
\begin{center}
$n = 2$ y $2 > 1$:
\\
$(1,1) : (1,1) \in R \wedge (1,1) \in R$
\\$(1,2) : (1,1) \in R \wedge (1,2) \in R$
\\$(1,3) : (1,2) \in R \wedge (2,3) \in R$
\\$(2,4) : (2,3) \in R \wedge (3,4) \in R$
\end{center}
$R^2 = \lbrace{(1, 1),(1, 2),(1, 3),(2, 4)}\rbrace$
\\
\\
\begin{center}
 $n = 3$ y $3 > 1$:

$(1,1) : (1,1) \in R^2 \wedge (1,1) \in R$
\\$(1,2) : (1,1) \in R^2 \wedge (1,2) \in R$
\\$(1,3) : (1,1) \in R^2 \wedge (1,3) \in R$
\\$(1,4) : (1,3) \in R^2 \wedge (3,4) \in R$
\end{center}
$R^3 = \lbrace{(1, 1),(1, 2),(1, 3),(1, 4)}\rbrace$

\end{document}



